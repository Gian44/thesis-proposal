\subsection{Related Works}
\label{subsec:relatedworks}

In the field of UCTP, heuristic approaches have been studied for extensive potential towards providing valid and effective solutions with respect to constraints. Metaheuristics aim at establishing quick but effective solutions by approximating optimal schedules rather than executing exhaustive searches, which can sometimes be very computationally expensive.

A two-stage very popular heuristic strategy by Bong Chia Lih et al. \cite{Lih2018-km} gives the first stage to be the grouping of all courses that can be held at the same time, and the second stage to be the assignment of time slots and venues for these groups. Meanwhile, the approach lowers the complexity of this problem, and it has already been successfully applied using real university data and it can handle real-world constraints very efficiently. Similar to Zhang et al., \cite{Zhang2014-ak} an efficient greedy metaheuristic algorithm handles room and time-slot assignments. Thus, the presented approach, emphasizing the predefined constraints, can meet the required flexibility by these timetables and institutional requirements.

Other metaheuristic methods applied in solving UCTP include Particle Swarm Optimization (PSO). PSO is another optimization technique that is increasingly gaining acceptance in solving UCTP problems. The concept was derived from the social behavior a swarm of birds or of fish follows. This makes it possible to adopt the efficient exploration of large search spaces. Oswald and Durai \cite{Oswald_C2013-zo} proposed a hybrid PSO with strategies for improving search applied particularly for UCTP. This was well balanced between exploration and exploitation so the robust timetabling solutions in reference to Oswald et al 2013. Similarly, Chen and Shih \cite{Chen2013-cp} created a constriction PSO model integrated with local search strategies to enhance both the gain in generation speed of solutions and the quality of the solution obtained. It has shown the effectiveness of PSO in solving challenging scheduling problems.

Multi-Swarm Particle Swarm Optimization (MSPSO) is an advanced variant of the standard PSO algorithm, aimed at enhancing performance in dynamic and multimodal optimization problems by dividing the population into multiple sub-swarms. Tim Blackwell and Jürgen Branke, in multiswarm PSO, incorporate mechanisms appropriate to dynamic optimization environments. \cite{Blackwell2006-ms} This methodology, while trying to maintain diversity over multiple swarms, focuses on keeping different swarms robustly tracking changing optima. It incorporates exclusion where swarms do not collapse to the same peak, as each pair of swarms within a predefined exclusion radius have weaker swarms reinitialized. It helps different swarms in this way to explore the different regions of the search space. In addition, anti-convergence occurs when all swarms converge; the least-fit swarm is reinitialized to search for emerging or unexplored peaks, ensuring adaptability in changing environments \cite{Blackwell2006-ms}. This algorithm further incorporates quantum particles which are randomly dispersed within a determined distance of the best known position inside the swarm. These maintain intraswarm diversity by facilitating the ability to quickly react and follow shifts in peeks. Combining these concepts permits MSPSO to balance global searches and local explorations efficiently with its performance surpassing PSO over dynamic multimodal landscapes \cite{ParticleSwarms2008}.

Generalizing the metaheuristic methods embraced here, such as greedy algorithms, adaptive genetic strategies, and PSO-based techniques, indicate that these are feasible for solving UCTP. In fact, these methods are not a global optimality guarantee; however, speed and flexibility in an ideally minimal number of computations make them highly valuable tools for educational institutions, especially for large-scale tasks where a large number of constraints are to be balanced.

\end{subs}