\section{Introduction}
\label{sec:introduction}

Academic institutions face the challenging task of effective resource management in their administrations without compromising the quality of the educational experience in this dynamic world of academia. \cite{Zhang2014-ak} Among the most complicated logistical problems that universities encounter is the so-called \textit{University Course Timetabling Problem} (UCTP). \cite{Arratia-Martinez2021-io} \cite{Oswald_C2013-zo} This problem deals with course-instructor-student assignment to particular time slots and rooms. It includes many constraints such as room capacities, faculty availability, and course prerequisites. \cite{Torres2021-ir} For solving these issues and establishing a standardized benchmark, the Second International Timetabling Competition (ITC-2007) proposed Track 3: Curriculum-Based Course Timetabling (CB-CTT). \cite{ITC2007-problem} This track defines the UCTP as the assignment of lectures to periods and rooms, subject to hard constraints that must be strictly satisfied (e.g., no room clashes) and soft constraints aimed at optimizing resource utilization and minimizing inconvenience (e.g., spreading lectures evenly). ITC-2007 provided datasets that have since become a critical benchmark for evaluating and comparing optimization algorithms for course timetabling. Many optimization techniques and algorithms have been applied to solve the CB-CTT problems. Some of the developed algorithms include heuristics, such as integer programming \cite{Lach2008-cbt} \cite{Burke2008-bcp} and metaheuristics-based algorithms, including local search based and hybrid algorithms. \cite{Bolaji2011-abc} \cite{Muller2007-itc} \cite{Abdullah2010-genetic}, \cite{Geiger2009-threshold}, \cite{Geiger2009-multicriteria}, \cite{Shaker2009-greatdeluge}, \cite{Lu2009-neighborhood}, \cite{Lu2010-adaptivetabu} The ideal schedule remains a challenge to prepare, especially for larger institutions, because manual methods lead to conflicts and inefficiencies.

Latest advanced computational approaches emerged to address these problems, and among them \textit{Swarm Intelligence} (SI) is one of the most promising directions. \cite{Algethami2021-mm} Among the algorithms of this group, there is an attention-grabber - \textit{Particle Swarm Optimization} (PSO), distinguished by adaptability and efficiency in searching large solution spaces. \cite{Chen2013-cp} \cite{Ali2014-mb}. The algorithm is represented by a set of particles, which are possible solutions; each particle moves in the search space, and its position changes according to individual and collective experiences.\cite{kennedy1995particle} \cite{Gunawan2008-ga} Still, despite many strengths, PSO fails at times to solve problems efficiently within UCTP, often failing to find the best solution under tight constraints. \cite{Oswald_C2013-zo}

Multi-Swarm Optimization (MSO) is designed to split up the main swarm into smaller, specialized sub-swarms that concurrently operate in exploiting different regions of the solution space simultaneously. \cite{Bacanin2022-multiswarm} \cite{MultiSwarm2004} It aims to improve the capabilities of PSO by focusing on issues such as search diversity and the global optimizing ability of the algorithm. \cite{XIA2018126} One of such particular modifications of the above-mentioned technique is MSPSO: \textit{Multi-Swarm Particle Swarm Optimization}, which dynamically varies the sub-swarms in the standard PSO in both exploration and exploitation for a good performance to provide capabilities of global and local search. \cite{MultiSwarm2004} \cite{Blackwell2006-ms} \cite{XIA2018126} To make this approach balanced, mechanisms for exploration and exploitation are integrated appropriately such that different solution domains would be explored in detail while optimizing the potential solutions in a balanced manner. \cite{Wang2023-ps} 

This paper introduces a new approach to solving the UCTP based on Multi-swarm PSO. MSPSO is specially designed for university course timetabling with diverse constraints, while search methods are efficient in finding optimal timetables. This approach promises to provide schedules that are conflict-free as well as resource-efficient and meet the specific needs of educational institutions. This effort aims to contribute towards scalable and realistic automation and optimization of timetabling, responding to one of the most complex problems in academic administration.

