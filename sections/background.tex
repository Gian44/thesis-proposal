\begin{subs}
    \subsection{Background of the Study}
    \label{subsec:background}
    \begin{itemize}
        \item[] \textbf{University Course Timetabling Problem (UCTP)}
        
        The \textit{University Course Timetabling Problem} (UCTP) is defined as a problem of making assignments for courses, instructors, and students to particular time slots and rooms according to numerous constraints, for example room capacities and instructor availability.\cite{Zhang2014-ak} \cite{Oswald_C2013-zo} \cite{Chen2013-cp} UCTP is a classic problem of academic optimization, and quality methods are demanded for efficient scheduling.\cite{Arratia-Martinez2021-io} \cite{Lih2018-km} \cite{Yang2017-ly} 
    
        \item[] \textbf{Heuristic Algorithms}
    
        \textit{Heuristic Algorithms} are heuristic methods to solve optimization problems that should find acceptable solutions within reasonable time. Problem-specific rules or strategies will be employed to search effectively in the solution space to provide a good enough solution rather than guaranteeing optimality in the whole problem space. For example, greedy algorithms, local search, and evolutionary heuristics have been very popular for applications in scheduling problems like UCTP \cite{Zhang2014-ak} \cite{Lih2018-km}.
    
        \item[] \textbf{Swarm Intelligence}
        
        \textit{Swarm Intelligence} refers to bio-inspired algorithms based on the collective behavior of animals such as ants, bees, or birds. \cite{Gao2024-apso} \cite{Fallahi2022-qpso} It is applied to optimization problems because it can efficiently and adaptively search very large solution spaces. \cite{Oswald_C2013-zo} \cite{Gao2024-apso}
    
        \item[] \textbf{Particle Swarm Optimization (PSO)}
        
        \textit{Particle Swarm Optimization (PSO)} is the best known among SI algorithms. \cite{Liu2017-clqpso} It simulates the motions of particles, which denote solutions, in the search space and updates them based on individual and collective experiences. \cite{Ali2014-mb} \cite{Zhan2009-apso} PSO performs very well on many optimization problems, such as UCTP, but still struggles with tricky constraints. \cite{Chen2013-cp} \cite{Oswald_C2013-zo}
    
        \item[] \textbf{Multi-Swarm Optimization (MSO)} 
    
        \textit{Multi-Swarm Optimization (MSO)} is an advanced extension of swarm intelligence algorithms. The basic idea in MSO is to divide the primal swarm into multiple sub-swarms that scan concurrently the different areas of the solution space. A concurrent search makes the algorithm have a better opportunity to discover global optima with effective balancing between exploration and exploitation.\cite{Bacanin2022-multiswarm}. \cite{MultiSwarm2004} 
    
        \item[] \textbf{Multi-swarm Particle Swarm Optimization (MPSO)} 
        
        \textit{Multi-Swarm Particle Swarm Optimization (MPSO)} is a PSO algorithm developed to overcome the lack of diversity of search and local optima avoidance in the process of complex optimization problems, by splitting the main swarm into a considerable number of subswarms that explore different regions in parallel with the solution space, hence, it promotes diversity and the likelihood of finding the global optima. \cite{XIA2018126} \cite{Blackwell2006-ms} \cite{Liu2023-pso} The difference gives the algorithm an excellent improvement in searching for such solutions for complex problems, such as the UCTP, which balanced exploration and exploitation would provide. 
    
    \end{itemize}