\section{Scope and Limitation}
\label{sec:scopeandlimitation}

This study deals with solving the University Course Timetabling Problem (UCTP) with the application of the Multi-Swarm Particle Swarm Optimization algorithm. The use of the ITC2007 Track 3 dataset as a major benchmark is incorporated within the study for assessing the performance of the proposed model, thereby allowing the comparison with previously established methods within the area of study to be maintained with a degree of consistency. This research studies the use of MSPSO for timetabling, optimizing a set of schedules based on different constraints: rooms, course prerequisites, and faculty assignments to provide quality timetables.

Another major aspect is its deployment as a web-based application in which the MSPSO-based solution has been designed and deployed as an interface, hence giving institutions the facility of practically creating timetables in a manner that scheduling conflicts can be minimized, and the available resources maximally utilized.

The only drawback is that the study has been done only on the ITC2007 Track 3 dataset, and it might limit the generality of results for other datasets or real-world problems. Furthermore, the performance of the algorithm is assessed in predefined constraints without taking into account dynamic changes such as last-minute course additions or room reassignments. Still, the work forms a very solid base for the application of MSPSO to timetabling and other optimization problems.
