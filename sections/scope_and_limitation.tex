\section{Scope and Limitation}
\label{sec:scopeandlimitation}

This paper introduces a new optimization approach from the perspective of MPSO for the UCTP. It examines some of the critical constraints in intense scrutiny and in depth, like room availability, course prerequisites, and faculty course assignments. The MPSO-based model aims to improve the efficiency of schedules across several constraints within institutions. In order to ensure reliability and adaptability in various contexts and environments, real-time validation is carried out on the model. Lastly, the evaluation of the MPSO model will also show comparison with existing algorithms in order to meet the preferences of faculties as well as serve as a more holistic benchmark.

However, some limitations have been encountered in this study. Even though the system is targeting to make resource usage up to the fullest as well as making the conflict avoidance probability to be at minimum, it does not comprise those dynamic changes which exist on the process of scheduling such as changing the courses at the last minute or rooms available suddenly. The model will be tested strictly within predefined constraints and data originating from multiple institutions, which can limit the general applicability of obtained results in different educational contexts. Therefore, although the study is focused on satisfying hard constraints, the subjective aspects of scheduling-for example, preferences of faculty members-might not be fulfilled fully by the final solution, but they will be taken into account for benchmarking purposes.
