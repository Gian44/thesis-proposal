\section{Significance of the Study}
\label{sec:significance}

The importance of this paper is developing a new approach for solving the University Course Timetabling Problem using Multi-Swarm Particle Swarm Optimization. To be more specific, because numerous methods have been widely applied to solve UCTP, MSPSO is the first approach that explores more extensive areas of solutions through the use of more than one sub-swarm. This new technique aims to generate high-quality, conflict-free timetables that respect necessary constraints such as room capacities, course prerequisites, and faculty schedules, thereby advancing the field of university scheduling.

Besides the academic value, this research opens up possibilities for wider applications of MSPSO in fields requiring efficient scheduling and resource management. With the establishment of MSPSO as an efficient strategy for UCTP, this study opens avenues for its use in other complex scheduling scenarios related to healthcare, transportation, and manufacturing. These contexts are critical where optimum planning and resource utilization must occur. The research outcomes from this study might stimulate new studies and further applications in MSPSO for diverse industry fields.

Therefore, this paper not only develops a novel solution for scheduling at the university level but also lays down a foundation that may help in further studying MSPSO as a broad tool for automated scheduling in multiple industries.