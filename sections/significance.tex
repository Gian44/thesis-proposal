\section{Significance of the Study}
\label{sec:significance}

The importance of this study is that it introduces the multi-swarm structure into the University Course Timetabling Problem (UCTP). MSPSO enhances how the solutions are generated by working together in parallel with several swarms to be able to provide a much stronger approach to possibly avoiding being caught in the trap of getting local optima as compared with standard methods for getting high-quality timetables against institutional constraints.

The study further underscores practicality by implementing the MSPSO solution as a web application. This shall be user-friendly to facilitate easier generation of timetables on the part of institutions, diminish potential scheduling conflicts, and optimize resource use such as rooms and faculty. Through its actual implementation, this approach will ensure it is ready for the real world, ready for administrators and other stakeholders.

It serves as a groundwork for future investigations on multi-swarm optimization for UCTP and other application areas. Based on the above proofs, the ability of MSPSO makes it suitable as a good foundational research work which may help others in doing future research on these areas of problem-solving as presented above.

