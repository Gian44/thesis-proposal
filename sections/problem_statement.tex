\section{Statement of the Problem}
\label{sec:problemstatement}

The UCTP problem is quite complex and key for any institution of learning because it considers assigning courses to specific rooms and time slots while considering constraints such as room capacities, course prerequisites, and faculty courses. \cite{Arratia-Martinez2021-io} To this date, most universities still consider non-optimal manual or heuristic methods, and the inefficiencies include uneven teaching loads and a mismatch between qualifications and requirements in course teaching. \cite{Oswald_C2013-zo} \cite{Gunawan2008-ga} While the complexity of the problem is increased, it happens to be highly infeasible to find the best solutions, especially when there are conflicting constraints that need to be fulfilled. 

The existing work \cite{Oswald_C2013-zo} \cite{Ali2014-mb} \cite{Chen2013-cp} used PSO-based techniques in solving UCTP but highly developed approaches are required to be found due to the complexity of the problem, such as MPSO, this improves the original PSO technique by splitting the swarm into many sub-swarms that simultaneously start exploring different regions of solution space. This allows for a higher degree of diversity in search and avoids getting trapped in the local optimum for a better possibility of finding the global optimum.

This paper focuses on the development of a solution for UCTP using MPSO, that is, to explore the vast solution space with much greater efficiency than before, paying attention to constraints like room capacities, course prerequisites, and scheduling conflicts. In terms of metrics like balance of workloads, course-faculty alignment, and computational efficiency, this proposed approach is supposed to bring about improvements in quality and fairness in the courses scheduled within institutions.
