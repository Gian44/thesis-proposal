\begin{abstract}
    The University Course Timetabling Problem (UCTP) is considered a difficult academic problem because the task is to assign courses, instructors, and students to time slots and rooms subject to several constraints. The authors of this paper present a novel approach by Multi-Swarm Particle Swarm Optimization (MSPSO) to split up the optimization into sub-swarms to increase the exploration and exploitation abilities. The methodology involves data preprocessing, optimization by MSPSO, and performance evaluation by using ITC2007 benchmark datasets. Results show the efficiency and consistency of MSPSO in producing feasible timetables at different levels of complexity. Additionally, a web-based interface was implemented for practical application, enabling real-time generation and validation of timetables. This research work shows that MSPSO has the potential for scalable and efficient solutions to the challenges of timetabling.
\end{abstract}